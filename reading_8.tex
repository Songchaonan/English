\documentclass{article}
\usepackage{float}
\usepackage{indentfirst}
\usepackage{graphicx}
\setlength{\parindent}{2em}
\setlength{\parskip}{1ex plus 0.5ex minus 0.2ex}
\selectfont
\title{The Internet of Things}
\author{Chaonan Song}
\begin{document}
\maketitle
\section{Describe the picture}
    \par As we can see from the picture,two persons are discussing about reading. While to our amusement, the boy says his favorite book is Facebook. The picture seems to be humorous and ridiculous but thought-provoking on deep thoughts, which intends to inform us that the social network has exerted an important impact on our daily reading.
\begin{figure}[H]
  \centering
  % Requires \usepackage{graphicx}
  \includegraphics[width=0.8\textwidth]{1.jpg}
  \caption{"I love reading. I read about 3 hours a day. My favorite book is Facebook."}
  \label{fig:1}
\end{figure}
\section{The impact of social networking websites on reading}
    \par Opinions vary when it comes to the impact of social networking  websites on reading. Some people insist that social network provides large collections of information at a tremendous speed and stimulates their reading interest. On the contrary, other people claim that it is a common phenomenon that a host of youngsters spend so much time reading on social networks that they don't have adequate opportunities or time to read traditional books.
\section{Show you own opinion}
    \par There is a saying goes like this, "Every coin has its two sides" . So there is no surprise that there are different opinions about the impact of social networking websites on reading. However, as a college student, I am convinced that it is necessary for us to read on social networking websites,  but it is of greater necessity for us to read traditional books, because social networks are just our tools and never can we depend on it in everything.
\end{document}

